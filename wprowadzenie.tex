\chapter*{Wprowadzenie}

Współczesne rynki finansowe, w tym rynek walutowy Forex, charakteryzują się ogromną zmiennością, a do ich skutecznej analizy wykorzystuje się coraz większe ilości danych. W obliczu rosnącej złożoności analizy rynków inwestorzy instytucjonalni coraz częściej sięgają po zaawansowane narzędzia technologiczne, aby skuteczniej podejmować decyzje inwestycyjne. W ostatnich latach szczególną uwagę zyskały systemy wsparte sztuczną inteligencją (SI), które oferują nowe możliwości w zakresie analizy danych, prognozowania trendów rynkowych oraz automatyzacji procesu handlu. 

Celem tej pracy magisterskiej jest zaprojektowaie i wdrożenie w środowisku testowym prototypu w pełni autonomicznego systemu handlującego na rynku Forex, który w procesie podejmowania decyzji inwestycyjnych wykorzystuje metody sztucznej inteligencji. Prototyp ma na celu zademonstrowanie możliwości najnowszych technologi w autonomicznym handlu oraz sprawdzenie jego efektywności w warunkach rzeczywistych. W ramach pracy przeanalizowane zostaną różne podejścia do budowy autonomicznych systemów handlowych: oparte na sztucznej inteligencji, analizie technicznej oraz handlu wysokich częstotliwości.

Praca została podzielona na kilka etapów. Pierwszym krokiem jest zrozumienie podstawowych zasad funkcjonowania rynku Forex, jak również metod skutecznej spekulacji. Następnie przeprowadzona została analiza literatury, która pozwoliła wyodrębnić najbardziej obiecujące algorytmy oraz standardy rynkowe odnośnie projektowania tego typu systemów. Kolejnym etapem było projektowanie systemu, które obejmuje wybór odpowiednich algorytmów sztucznej inteligencji oraz architektury systemu. Następnie przeprowadzona została implementacja prototypu oraz testy wsteczne działania w symulowanych warunkach rynkowych. Na zakończenie, wyniki działania systemu zostaną poddane szczegółowej analizie, aby ocenić jego skuteczność, ograniczenia oraz możliwości dalszego rozwoju.